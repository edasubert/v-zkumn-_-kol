\documentclass[text.tex]{subfiles}

\begin{document}
\section*{Computation}
\addcontentsline{toc}{section}{Computation}
It should not be a surprise that most of the analysis is done with a computer programs. These programs were not yet discussed since they are a literal transcriptions of the described algorithms. There are two concepts that make it possible: precise arithmetic in $\field$ and object-oriented programming. 
\subsection{Precise arithmetic in $\field$}
Nearly every calculation in the algorithms is inside the number field $\field$. It is possible to implement a custom class that will enable precise calculations. Since $\beta$ is quadratic the field has the following simple form.
$$\field = \left\{\left. \frac{a + b\beta}{c} \right|a,b\in \mathbb{Z},\; c\in\NN \right\}$$
Therefore each number $x\in\field$ can be represented by three integers and integer arithmetic is precise. Several operations are listed here. Particularly useful is the \texttt{simplify} function which makes sure that $a,\;b,\;c$ do not overflow. (\texttt{gcd} returns the greatest common divisor)

\begin{algorithm}
	%\SetKwFunction{algo}{algo}\SetKwFunction{struct}{}
	\SetKwFunction{algo}{algo}\SetKwFunction{add}{operator $+$ }
	\SetKwFunction{algo}{algo}\SetKwFunction{mul}{operator $*$ }
	\SetKwFunction{algo}{algo}\SetKwFunction{div}{operator $/$ }
	\SetKwFunction{algo}{algo}\SetKwFunction{com}{operator $<$ }
	\SetKwFunction{algo}{algo}\SetKwFunction{sim}{simplify}
	\SetKwProg{myalg}{Algorithm}{}{}
	
	%\setcounter{AlgoLine}{0}
	%\SetKwProg{myproc}{Struct}{}{}
	%\myproc{\struct{\textnormal{p}}}{
	%\nl int a\;
	%\nl int b\;
	%\nl int c\;
	%}
	
	\setcounter{AlgoLine}{0}
	\SetKwProg{myproc}{Function}{}{}
	\myproc{\add{\textnormal{p,q}}}{
	\nl p.a = p.a*q.c + q.a*p.c\;
	\nl p.b = p.b*q.c + q.b*p.c\;
	\nl p.c = p.c * q.c\;
	\nl \KwRet p\;}
	
	%\setcounter{AlgoLine}{0}
	%\SetKwProg{myproc}{Function}{}{}
	%\myproc{\mulint{\textnormal{p,i}}}{
	%\nl p.a = p.a*i\;
	%\nl p.b = p.b*i\;
	%\nl p.c = p.c\;
	%\nl \KwRet p\;}
	
	\setcounter{AlgoLine}{0}
	\SetKwProg{myproc}{Function}{}{}
	\myproc{\mul{\textnormal{p,q}}}{
	\nl p.a = p.a*q.a - p.b*q.b\;
	\nl p.b = p.b*q.a + p.a*q.b + 4*p.b*q.b\;
	\nl p.c = p.c * q.c\;
	\nl \KwRet p\;}
	
	\setcounter{AlgoLine}{0}
	\SetKwProg{myproc}{Function}{}{}
	\myproc{\div{\textnormal{p,q}}}{
	\nl p.a = (p.a*q.a + p.b*q.b + 4*p.a*q.b)*q.c\;
	\nl p.b = (p.b*q.a - p.a*q.b)*q.c\;
	\nl p.c = (q.a*q.a + 4*q.a*q.b + q.b*q.b)*p.c\;
	\nl \KwRet p\;}
  
  \setcounter{AlgoLine}{0}
	\SetKwProg{myproc}{Function}{}{}
	\myproc{\com{\textnormal{p,q}}}{
  \nl A = p.a*q.c+p.b*q.c*2 - q.a*p.c-q.b*p.c*2\;
  \nl B = q.b*p.c - p.b*q.c\;
  \nl \If{sign(A)*A*A $<$ 3*sign(B)*B*B}{
  \nl \KwRet true\;}
	\nl \KwRet false\;}
  
  \setcounter{AlgoLine}{0}
	\SetKwProg{myproc}{Function}{}{}
	\myproc{\sim{\textnormal{p}}}{
  \nl cache = 0\;
  \nl \If{((cache = gcd(a,b)) != 0) $\&\&$ ((cache = gcd(cache,c)) != 0) $\&\&$ (cache $>$ 1 )}{
  \nl p.a = p.a/cache\;
  \nl p.b = p.b/cache\;
  \nl p.c = p.c/cache\;}
	\nl \KwRet p\;}
  
	\caption{Precise arithmetic inside $\field$.}
\end{algorithm}

\subsection{Object-oriented programming}
Many objects were implemented to represent various structures used in the algorithms: \texttt{Cpoint}, \texttt{CpointSet}, \texttt{CDeloneSet}, \texttt{CVoronoiCell} and more. Such abstraction truly allows a literal transcription of the algorithms as they are explained in the text. 

If you are interested in diving into the code yourself, every program used to make this text is available online:
\url{https://github.com/edasubert/quasicrystal}

The algorithms a programs for the golden ratio variant of quasicrystal are described in great detail in \cite{magister}. 
\end{document}
