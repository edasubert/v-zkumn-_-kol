\documentclass[text.tex]{subfiles}

\begin{document}

\section{Division of window}%========================================================================================
Previous section has established that for each point of the quasicrystal, the shape of the associated voronoi polygon is only influenced by the points of the quasicrystal that are closer then $2L\cdot\hat{R}_c$, where $L$ is the largest distance for a given window. 

In this section we describe the algorithm to divide one-dimensional window to parts by the same corresponding words. That is vital for two-dimensional quasicrystal analysis.

\begin{theorem}
\label{the:spaces:iteration}
Function ${(f^\Omega)}^n$ denotes the $n$-th iteration of the stepping function of the quasicrystal $\quasi{\Omega}$. Set $D_n = \{z_1 < z_2 < \dotsb < z_{m-1}\}$ contains all discontinuities of ${(f^\Omega)}^n$, $z_0 = c$ and $z_m = d$.\\
Then $(\forall i \in \widehat{m}\cup\{0\})(\forall {\left(y_l^\Omega\right)}' ,{\left(y_k^\Omega\right)}' \in (z_i, z_{i+1}))$ are words $t_l^\Omega t_{l+1}^\Omega \dotso t_{l+n-1}^\Omega$ and $t_k^\Omega t_{k+1}^\Omega \dotso t_{k+n-1}^\Omega$ the same.
\end{theorem}

\begin{remark}
In other words the Theorem \ref{the:spaces:iteration} states that the discontinuities of the $n$-th iteration of the stepping function divide the window into intervals of images of the points of the quasicrystal after which the same sequence of distances of the length $n$ follow.
\end{remark}

\begin{theorem}
\label{the:complexity}
Let $\Omega = [c,c+\ell)$.\\
If $\ell \notin \ring$ then $\mathcal{C}_{\ell}(n) = 2n+1,\, \forall n\in\mathbb{N}$. \\
If $\ell \in \ring$ then $\exists^1 k \in \mathbb{N}$ such that $\left({(f^\Omega)}^{k}(a^\Omega) = b^\Omega\right)$ or $\left({(f^\Omega)}^{k+1}(b^\Omega) = a^\Omega\right)$ and 
$$\mathcal{C}_{\ell}(n) = \left\{
	\begin{array}{l l}
		2n+1	&	\forall n\leq k \\
		n+k+1	&	\forall n > k
	\end{array}
	\right.
$$
\end{theorem}

\begin{theorem}
\label{the:sameSpaces}
Let $\Omega = [c,c+\ell)$.
$$\mathcal{D}_n = \left\{ \ell\,\left|\, \ell\in \left(\frac{1}{\beta},1\right] \,\wedge\, \mathcal{C}_\ell(n) < 2n+1 \right.\right\}$$
Then elements of $\mathcal{D}_n$ divide interval $I \vcentcolon= \left(\frac{1}{\beta},1\right]$ into finite amount of disjoint sub-intervals $(I_m)_{m\in\hat{N}}$ such that $\mathcal{L}_{\ell_1}(n) = \mathcal{L}_{\ell_2}(n)$ $\forall \ell_1, \ell_2 \in I_m,\, \forall m\in\hat{N}, \forall n\in\NN$.
\end{theorem}

\begin{remark}
Please note that $\mathcal{D}_n$ from theorem \ref{the:sameSpaces} divides base windows into sets of same language whereas $D_n$ from theorem \ref{the:spaces:iteration} divides specific window into intervals by the sequences of points that follow.
\end{remark}



\begin{figure}[h]
\centering
\caption{Graph of stepping function for quasicrystal $\quasi{\Omega}$ where $\Omega = [c,d),\, c=0,\, d=12-3\beta$. $C= 2\beta-1$, $D= \beta$ and $E= \beta-1$.}
\label{img:steppingFunction}
\begin{tikzpicture}[scale=1.2]
\draw [<->] (0,5.3) -- (0,0) -- (5.3,0);

\draw [thick,*-o,shorten <=-3pt,shorten >=-3pt] (0,1.674682453) 				-- (3.325317548,5);
\draw [thick,*-o,shorten <=-3pt,shorten >=-3pt] (3.325317548,0.4246824527)	-- (4.575317547,1.674682453);
\draw [thick,*-o,shorten <=-3pt,shorten >=-3pt] (4.575317547,0) 				-- (5,0.4246824527);

\draw [dotted] (0,0.4246824527) -- (5.3,0.4246824527);
\draw [dotted] (0,1.674682453)  -- (5.3,1.674682453);
\draw [dotted] (0,5) -- (5.3,5);

\draw [dotted] (3.325317548,0) -- (3.325317548,5.3);
\draw [dotted] (4.575317547,0) -- (4.575317547,5.3);
\draw [dotted] (5,0) -- (5,5.3);

\node [above left] at (3.325317548/2,3.3373412265) {$D$};
\node [above left] at (3.9503175475,1.04968245285) {$C$};
\node [above left] at (4.7876587735,0.4246824527/2) {$E$};

\node [below left] at (0,0) {$0$};
\node [below] at (5,0) {$d$};
\node [left] at (0,5) {$d$};
\end{tikzpicture}
\end{figure}

The algorithm uses the stepping function of a quasicrystal. As is apparent from Figure \ref{img:steppingFunction} and Theorem \ref{the:ffff} DOPLNIT, stepping function is piece wise linear and after points of quasicrystal corresponding to one linear segment follows the same distance to the next point of the quasicrystal. Alternatively all the points of the sequence of the quasicrystal $y_n$ whose images $y_n'$ are in a single segment of linearity, have the same corresponding letter in the word of the quasicrystal. That is precisely what the algorithm uses. 

First only non-singular windows are considered.

\paragraph{Algorithm definition}
Algorithm receives as an input an interval $\Omega = [c,d)$ representing the window of the quasicrystal and $n\in\NN$ representing the desired length of the words.

As an output algorithm provides the division of $\Omega$ into disjunt intervals $[\omega_0,\omega_1)$, $[\omega_1,\omega_2), \dots, [\omega_{m-1},\omega_m)$ such that $\omega_0 = c$ and $\omega_m = d$.
$$\left(\forall y_j^\Omega,y_k^\Omega\in\left(y_n^\Omega\right)_{n\in\ZN} \right)\left(\forall i\in\widehat{m-1}\right): \left(\left({y_j^\Omega}\right)',\left({y_k^\Omega}\right)' \in [\omega_i,\omega_{i+1})\right) \Rightarrow \left(\left(t_n^\Omega\right)_j^{j+n} = \left(t_n^\Omega\right)_k^{k+n}\right)$$

The division is acquired by recursion. 

For $n=1$ is the division already known.
$$m=3, \omega_1 = a^\Omega, \omega_2 = b^\Omega$$

For $n>1$ is the division found from the division for $n-1$.
Intervals 
$[\omega_0^{n-1},\omega_1^{n-1})$, $[\omega_1^{n-1},\omega_2^{n-1}), \dots, [\omega_{m-1}^{k-1},\omega_{k}^{n-1})$ denote the division for $n-1$.

For each interval $[\omega_i^{n-1},\omega_{i+1}^{n-1})$ the stepping function image is evaluated. 
$$f^\Omega\left([\omega_i^{n-1},\omega_{i+1}^{n-1})\right) = [f^\Omega(\omega_i^{n-1}),f^\Omega(\omega_{i+1}^{n-1}))$$
Then the image is divided by the points $a^\Omega$ and $b^\Omega$. If one or both of these points are inside the image, it gets divided into two or three disjunct intervals. 

After all intervals for $i\in \widehat{k-1}$ are processed, all images or their divisions are sorted and denoted $[\omega_0,\omega_1), [\omega_1,\omega_2), \dots, [\omega_{m-1},\omega_m)$.

It may also be desirable to not only acquire the division of the window by the same words, but to also acquire the words themselves. That is done by a simple modification of the described algorithm. Each interval is marked with the corresponding letter $A$, $B$, $C$, $D$ or $E$ at the beginning of the recursion. While dividing the image of the interval by the points $a^\Omega$ and/or $b^\Omega$, the mark is appended by an appropriate letter.

\begin{figure}
\begin{center}
\begin{tikzpicture}

\coordinate (xc) at (0,0);
\coordinate (xa) at (15*0.3301270191,0);
\coordinate (xb) at (15*0.4641016151,0);
\coordinate (xd) at (15*0.5980762115,0);

\node [above] at (xc) {$c$};
\node [above] at (xa) {$a^\Omega$};
\node [above] at (xb) {$b^\Omega$};
\node [above] at (xd) {$d$};

% guide lines
\draw [dashed,ultra thin, opacity=0.6] (xc) -- (xa) -- (15*0.598076,-1) -- (15*0.267949,-1) -- cycle;
\draw [dashed,ultra thin, opacity=0.6] ($(xc)+(0,-1)$) -- ($(xa)+(0,-1)$) -- (15*0.598076,-2) -- (15*0.267949,-2) -- cycle;
\draw [dashed,ultra thin, opacity=0.6] ($(xc)+(0,-2)$) -- ($(xa)+(0,-2)$) -- (15*0.598076,-3) -- (15*0.267949,-3) -- cycle;
\draw [dashed,ultra thin, opacity=0.6] ($(xc)+(0,-3)$) -- ($(xa)+(0,-3)$) -- (15*0.598076,-4) -- (15*0.267949,-4) -- cycle;

\draw [dashed,ultra thin, opacity=0.6] (15*0.330127, 0) -- (15*0.464102, 0) -- (15*0.267949,-1) -- (15*0.133975,-1) -- cycle;
\draw [dashed,ultra thin, opacity=0.6] (15*0.330127,-1) -- (15*0.464102,-1) -- (15*0.267949,-2) -- (15*0.133975,-2) -- cycle;
\draw [dashed,ultra thin, opacity=0.6] (15*0.330127,-2) -- (15*0.464102,-2) -- (15*0.267949,-3) -- (15*0.133975,-3) -- cycle;
\draw [dashed,ultra thin, opacity=0.6] (15*0.330127,-3) -- (15*0.464102,-3) -- (15*0.267949,-4) -- (15*0.133975,-4) -- cycle;

\draw [dashed,ultra thin,opacity=0.6] (15*0.464102, 0) -- (15*0.598076, 0) -- (15*0.133975,-1) -- (0,-1) -- cycle;
\draw [dashed,ultra thin,opacity=0.6] (15*0.464102,-1) -- (15*0.598076,-1) -- (15*0.133975,-2) -- (0,-2) -- cycle;
\draw [dashed,ultra thin,opacity=0.6] (15*0.464102,-2) -- (15*0.598076,-2) -- (15*0.133975,-3) -- (0,-3) -- cycle;
\draw [dashed,ultra thin,opacity=0.6] (15*0.464102,-3) -- (15*0.598076,-3) -- (15*0.133975,-4) -- (0,-4) -- cycle;


\path [fill,opacity=0.3,thin] (xc) -- (xa) -- (15*0.598076,-1) -- (15*0.267949,-1);

\path [fill,opacity=0.3,thin] (15*0.330127,-1) -- (15*0.464102,-1) -- (15*0.267949,-2) -- (15*0.133975,-2);
\path [fill,opacity=0.3,thin] (15*0.133975,-2) -- (15*0.267949,-2) -- (15*0.535898,-3) -- (15*0.401924,-3);
\path [fill,opacity=0.3,thin] (15*0.401924,-3) -- (15*0.464102,-3) -- (15*0.267949,-4) -- (15*0.205771,-4);

\foreach \x in {0,...,4}
{
\draw [|-|] ($(xc)+(0,-\x)$) -- ($(xd)+(0,-\x)$);
\draw [fill] ($(xc)+(0,-\x)$) circle [radius=0.1];
\draw [fill] ($(xa)+(0,-\x)$) circle [radius=0.1];
\draw [fill] ($(xb)+(0,-\x)$) circle [radius=0.1];
\draw [fill=white] ($(xd)+(0,-\x)$) circle [radius=0.1];
}

\end{tikzpicture}
\end{center}
\label{pic:iteration}
\caption{Iteration of the stepping function $f^{\Omega}$ where $|\Omega| = \frac{3\beta-10}{2}$. Dashed lines show the exchange of intervals of the stepping function and the gray area shows progression of division for one interval.} 
\end{figure}


\end{document}
