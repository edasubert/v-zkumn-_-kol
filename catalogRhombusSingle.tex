\documentclass[text.tex]{subfiles}

\begin{document}
\section{Catalog of voronoi polygons for a single rhombus window}
This section describes the algorithm for generating all voronoi polygons in a quasicrystal with a rhombus window. Unless otherwise stated, in this section quasicrystal always means two-dimensional quasicrystal and window always means rhombus window. The key components from previous sections are:

\begin{enumerate}
\item only the points of the quasicrystal that are close enough determain the shape of the voronoi polygon
\item finite section of the quasicrystal is easy to generate from two one-dimensional quasicrystals
\item language $\mathcal{L}_{\ell}(n)$ is finite and easy to generate
\end{enumerate}

The algorithm is then straight forward:

\paragraph{Algorith definition} The algorithm receives as an input a rhombus window. As an output it returns a list of voronoi polygons found in the quasicrystal coresponding to the given window.

The largest distance within the coresponding one-dimensional quasicrystal is denoted by $L$. Steps are described in more detail bellow.

\begin{enumerate}
\item evaluate $L\cdot\hat{R}_c$ covering radius of the quasicrystal
\item determine the lenght of a word $n$ sufficient to cover a circle of radius $2L\cdot\hat{R}_c$
\item generate the language $\mathcal{L}_{\ell}(n)$
\item generate finite sections of the quasicrystal for each pair of the words form the language $\mathcal{L}_{\ell}(n)$ such that each finite section contains origin
\item construct a voronoi polygon for the origin for each finite section
\item filter duplicate voronoi polygons
\end{enumerate}

To justify that the algorithm finds all voronoi polygons for a given window consider, that the shape of each voronoi polygon is determined by the points of the quasicrystal, that are distant at most $2L\cdot\hat{R}_c$ from the center of the polygon. In other words, the shape is only determined by a finite section of the quasicrystal. Each finite section of the quasicrystal with a rhombus window is described by two finite sections of a one-dimensional quasicrystal. Each finite section of a one-dimensional quasicrystal of length $n$ is described by a finite word of the quasicrystal, that are all present in the language $\mathcal{L}_{\ell}(n)$.

\subsection{Determine sufficient $n$}
Part of the algorithm that was not covered in detail is how to determine the length of a word sufficianet to cover a circle of radius $2L\cdot\hat{R}_c$. First a rhombus is circumscribed to the circle of the radius $2L\cdot\hat{R}_c$. The side of such rhombus is $4$ times larger than the circle radius. Then such $n$ has to be found that a finite section of one-dimensional quasicrystal corresponding to each word from $2L\cdot\hat{R}_c$ has to be at least as long as the side of the circumscribed rhombus. 
There are several approaches, two are described here. 

One way is to get the smallest distance $S$ for the one-dimensional quasicrystal and set $n = \left\lceil\frac{8L\cdot\hat{R}_c}{S}\right\rceil$. 

The second way is to start with $n=2$ and increase by $1$ until is $n$ sufficient. 

The second way takes more time to compute but produces better estimate, which will be desirable once analyzing quasicrystals with a general window.

\subsection{Diagram}
Following diagram might help with understanging the algorithm.
\end{document}